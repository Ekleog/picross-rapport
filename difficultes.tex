\newcommand\thread{\emph{thread}}
\newcommand\threads{\emph{threads}}
\newcommand\lifetime{\emph{lifetime}}
\newcommand\lifetimes{\emph{lifetimes}}
\newcommand\inl{\mintinline{rust}}

\section{Difficultés rencontrées}

\subsection{Rust}

Les premières et principales difficultés rencontrées proviennent du langage de
programmation choisi, Rust. En effet, nous l'avons choisi car, en ayant entendu
dire énormément de bien, nous nous sommes dit qu'il serait une bonne idée de
profiter d'un Projet d'Informatique pour apprendre le langage et le mettre en
pratique sur un cas concret.

En conséquence, nous avons découvert les difficultés que peut proposer rust
avant d'en découvrir les solutions simples, et avons dû nous battre contre des
mécanismes originalement conçus pour simplifier la vie.

\subsubsection{Borrow checker}

Le \emph{borrow checker} est une passe du compilateur qui vise à vérifier que
toute variable n'est accessible en écriture qu'à un endroit à la fois, et
uniquement si aucun autre endroit n'y a accès en lecture.

Ceci a pour utilité d'empêcher des \emph{race conditions}, en évitant que deux
\threads puissent écrire en même temps dans le même objet en mémoire, ou bien en
évitant qu'un \thread tente de lire un objet en cours de modification par un
autre.

Il sert également à éviter, par exemple qu'alors que l'on est en train d'itérer
sur un \inl!Vec<T>!.

Cependant, cela peut amener à des problèmes singuliers, dans certains cas. Le
problème pourrait souvent être résolu en refactorisant le programme, mais les
contraintes s'imposant sur un tel projet ne sont pas toujours compatibles avec
un code de propreté maximale.

Par exemple, le programme suivant, version simplifiée d'un cas réel qui nous est
apparu avec le Picross (les types ont été précisés pour plus de clarté mais ne
sont pas nécessaires)~:

\inputminted{rust}{BorrowChecker.rs}

Dans ce code, deux parties a priori indépendantes de \inl!v! sont accédées
simultanément~; mais le développeur sait qu'il est impossible qu'un problème
survienne.

Cependant, le compilateur rust n'étant pas capable de prouver cela, il renvoie
le message d'erreur suivant~:

\inputminted[breaklines]{rust}{BorrowChecker.log}

Ce problème est survenu en particulier lorsque nous avons tenté de parcourir en
même temps les spécifications et les cellules d'un \inl!Picross!, étant donné
que les deux sont stockés dans le même objet.

Nous avons dans ce cas particulier choisi de \inl!clone()!-er les variables. Une
autre solution aurait par exemple été de séparer les spécifications du
\inl!Picross! en lui-même.

\subsubsection{Lifetimes}

Les \lifetimes ont introduites afin de calculer à quel moment une variable
commence à vivre et à quel moment elle cesse de vivre. En effet, l'existence de
\emph{borrows} entraîne naturellement la question précise de quand une variable
n'est plus allouée. Si on prend l'exemple suivant~:

\inputminted{rust}{LifeTimeWorking.rs}

Cet exemple compile car il est assez simple pour que le compilateur arrive à
déterminer tout seul que la référence retournée a la même durée de vie que le
paramêtre, mais les lifetimes sont implicites. On pourrait annoter plus
précisément le code de \inl!held! en la définissant ainsi~:

\inputminted{rust}{LifeTimeAnnotated.rs}

Cette définition rend explicite le fait que le retour a la même durée de vie que
le paramètre, notée ici \inl!'a!.

Cependant, il y a des cas particuliers pour lesquels il est impossible de
définir une lifetime cohérente avec uniquement ces outils. Par exemple, ayant
repéré que le traitement plus lourd était de simplement générer toutes les
lignes possibles étant données les contraintes, nous avons d'abord décidé de
mémoizer les lignes calculées. Cependant, nous avons eu un souci de lifetimes~:

\inputminted[breaklines]{rust}{LifeTimeNotWorking.rs}

Ici, nous souhaitions retourner une référence vers l'élément dans \inl!cache!,
afin d'éviter une copie inutile. Cependant, il était impossible de le faire,
étant donné que \inl!cache.borrow_mut()! avait sa \lifetime qui terminait à la
fin de la \emph{closure}. Une solution, comme nous l'avons découvert après nous
être rendus compte qu'il était plus intelligent de simplement tout précalculer,
aurait été d'utiliser un \inl!Rc! à la place d'un \inl!Box!, un pointeur
intelligent à comptage de référence, qui permet de «~partager~»
l'\emph{ownership} d'un objet entre plusieurs points, et donc en particulier de
le sortir de la fonction.

\subsection{Optimisation future}

En exécutant le programme sur des cas de taille assez grande, nous avons pu
observer qu'un facteur limitant est la quantité de mémoire nécessaire allouée au
moment du précalcul de toutes les lignes possibles.

Une solution possible pour améliorer cela serait de ne pas stocker toutes les
lignes telles qu'on les utilisera, mais de simplement stocker le minimum
d'information pour pouvoir les reconstruire avec peu de calculs~: à la place
d'une ligne, stocker l'index de départ de chaque bloc de noirs consécutifs (donc
un \inl!Vec<usize>! contenant autant d'éléments que de blocs, à la place d'un
\inl!Vec<Cell>! contenant autant d'éléments que de cases dans la ligne ou
colonne).

Une autre optimisation serait, au fur et à mesure de l'avancée initiale des
combinaisons-exclusions, de supprimer les «~lignes possibles~» de la liste des
lignes précalculées. En effet, il n'est plus nécessaire de les conserver, étant
donné qu'on a déjà prouvé qu'elles ne pouvaient pas être réelles.
